\documentclass[12pt, a4paper]{report}
\usepackage{ucs}
\usepackage[T1]{fontenc}
\usepackage[utf8x]{inputenc}
\usepackage[ngerman]{babel}
\usepackage[pdftex]{graphicx}
\usepackage{a4wide}
\usepackage[onehalfspacing]{setspace}
\usepackage[left=2.5cm,right=2.5cm,top=2.5cm,bottom=2.5cm]{geometry}
\usepackage[marginal]{footmisc}
\usepackage{url}

\usepackage{longtable}

\usepackage{rotating}


%für die Kopfzeile
\usepackage{fancyhdr}
\pagestyle{fancy}
\fancyhf{}
\fancyhead[C]{\nouppercase{\leftmark}}
\fancyhead[R]{\thepage} 

%für römische Zahlen
\newcommand{\RM}[1]{\MakeUppercase{\romannumeral #1}}


\usepackage{multirow}
\usepackage{tabularx}

\usepackage{amsmath}


\usepackage{apager}


\usepackage{float}
\restylefloat{figure}


% Für Lit.-Verzeichnis folgenden Befehl im Terminal ausführen: bibtex Masterarbeit_Moritz_Hanke
\bibliographystyle{apager_dgps}


% Fürs Abkürzungsverzeichnis folgenden Befehl im Terminal ausführen: makeindex Masterarbeit_Moritz_Hanke.nlo -s nomencl.ist -o Masterarbeit_Moritz_Hanke.nls
\usepackage{nomencl}
\let\abk\nomenclature
\renewcommand{\nomname}{Abkürzungsverzeichnis}
\setlength{\nomlabelwidth}{.20\hsize}
\renewcommand{\nomlabel}[1]{#1 \dotfill}
\setlength{\nomitemsep}{-\parsep}
\makenomenclature 


\usepackage{Sweave}



\begin{document}

% Titelblatt
\thispagestyle{empty}

\begin{center}
\textbf{\LARGE{TITEL DER ARBEIT\\}} 
\end{center}

\begin{center}
\textbf{\Large{\\Masterarbeit}}
\end{center}
\begin{verbatim}

\end{verbatim}

\begin{figure}[htbp]
\begin{center}
\setkeys{Gin}{width=0.5\textwidth}
\includegraphics{Universitaet_Bremen.png}
\end{center}
\end{figure}

\begin{center}
\textbf{Fachbereich 3: Mathematik \\
Studiengang Medical Biometry/Biostatistics (M.Sc.)\\}
\end{center}
\begin{verbatim}

\end{verbatim}

\begin{flushleft}
\begin{tabular}{lll}
& & \\
& & \\
\textbf{Eingereicht von:} & & Hanke, Moritz \\
\textbf{Geboren am:} & & 07.08.1985 \\
\textbf{Matrikelnummer:} & & 2404575 \\
& & \\
& & \\
\textbf{Betreuung:} & & Prof. Dr. Iris Pigeot\\
& & Dr. Ronja Foraita \\
& & \\
& & \\
\textbf{Eingereicht am:} & & \today\\
& & \\
& & \\
\end{tabular}
\end{flushleft}





%Verzeichnisse
\pagenumbering{Roman}
\tableofcontents

\listoffigures

\listoftables

\printnomenclature 

\abk{GWAS}{Genome-wide association study}



% Begin der Arbeit
\chapter{Einleitung}
\pagenumbering{arabic}
bla bla bla bla bla bla bla bla bla bla bla bla bla bla bla bla bla bla bla bla bla bla bla bla bla bla bla bla bla bla bla bla bla bla bla bla bla bla bla bla bla bla bla bla bla bla bla bla bla bla bla bla bla bla bla bla bla bla bla bla bla bla bla bla bla bla bla bla bla bla bla bla bla bla bla bla bla \cite{kim_network-based_2013}




\chapter{Theorie}
\section{Netzwerke und Graphen}
\subsection{Graphen}
Mit einem Graphen $G=(V,E)$ wird ein abstraktes Objekt bezeichnet, das aus einer Menge $V=\{v_1,\dots,v_{N_V}\}$ an \textit{Knoten} und einer Menge $E=\{e_1,\dots,e_{N_E}\}$ an \textit{Kanten} besteht \cite{brandes2005graphfunda}. Die Kardinalitäten $N_V =|N|$ und $N_E=|E|$ geben die \textit{Ordnung} und die \textit{Größe} des Graphen an. Ist eine Kante $e_k \in E$ die Zuordnung zweier Knoten $v_i,v_j \in V$ eines Graphen $G=(V,E)$, gilt ${v_i,v_j} \in e_k$ und die Knoten $v_i$ und $v_j$ werden als \textit{inzident} zu dem Knoten $e_k$ und \textit{adjazent} zueinander bezeichnet. Für $e_k=\{v_i,v_j\}$ handelt es sich um ein \textit{ungeordnetes} Paar an Knoten, d.h. $\{v_i,v_j\}=\{v_j,v_i\}$. Graphen, die nur ungeordneten Knotenpaaren beinhalten, sind \textit{ungerichtet}. Dagegen bezeichnet $e=(u,v)$, dass es sich um ein \textit{geordnetes} Paar an Knoten handelt und die Kante $e$ vom \textit{Anfangsknoten} $u$ auf den \textit{Endknoten} $v$ gerichtet ist, womit $(u,v) \neq (v, u)$ gilt \cite{kolaczyk2009statistical}. Als \textit{gerichtete} Graphen werden Graphen bezeichnet, die nur geordnete Knotenpaare beinhalten. Der Graph $H=(W,F)$ ist ein \textit{Subgraph} von $G=(V,E)$ wenn $W \subseteq V$ und $F \subseteq E$ gilt. Die in dieser Arbeit verwendeten Graphen sind \textit{schlichte} Graphen, d.h. es gibt weder \textit{Schlingen} ($e_k={v_i,u_j}$ für $i=j$) noch mehrere Kanten zwischen zwei Knoten \cite{tittmann2011graphen}. Darüber hinaus werden nur \textit{endliche} Graphen berücksichtigt.\\
Die \textit{Adjazenzmatrix} $\textbf{A}=(a_{ij})_{N_V \times N_V}$, deren Zeilen- und Spaltenbeschriftung jeweils den Knoten aus $V$ entsprechen \cite{kolaczyk2009statistical}, gibt an, ob es eine Kante zwischen zwei Knoten gibt. Für einen ungerichteten Graphen\footnote{Werden von einem gerichteten Graph die Richtungen der Kanten entfernt, d.h. $(u,v)=\{u,v\}=(v,u)$, bleibt als "`Skelett"' ein ungerichteter Graph übrig, der die Adjazenzmatrix $\textbf{A}$ besitzt.} ist $\textbf{A}$ symmetrisch und es gilt 
\begin{align*}
A_{ij} = \begin{cases}
1, \qquad \text{wenn} \ \{v_i,v_j\} \in E,\\
0, \qquad \text{sonst.}\\
\end{cases}
\end{align*}
Handelt es sich um einen gerichteten Graphen, ist $\textbf{A}$ als
\begin{align*}
A_{ij} = \begin{cases}
1, \qquad &\text{wenn} \ (i,j),\\
-1, \qquad &\text{wenn} \ (j,i),\\
0, \qquad &\text{sonst}\\
\end{cases}
\end{align*}
definiert \cite{brandes2005graphfunda}.

%CLUSTER?
%Cliquen? (maximale?)
%E(G) oder E?
%DAG? <- kommt das bei Kim et al. vor?

\bibliography{Masterarbeit}


\begin{appendix}
% Die eidesstattliche Erklärung auf einer neuen Seite
\newpage
% Keine Nummerierung für diesen Teil
\section*{Eigenständigkeitserklärung}
% Keine Kopf- und Fußzeilen ausgeben
\thispagestyle{empty}
% Aber trotzdem ins Inhaltsverzeichnis aufnehmen
\addcontentsline{toc}{chapter}{Eigenständigkeitserklärung}
% Hier der offizielle Text der eidesstattlichen Erklärung
Hiermit versichere ich, dass ich die vorliegende Arbeit "`XXXXXXXXXXXXXXXXXXXXXXXXXXXXXXXXX XXXXXXXXXXXXXXXXXXXXXXXXXXXXXXXXXXXXXXX XXXXXXXXXXXXXXXXXXXXXXXXXXXX XXXXXXXXXXXXXXXX"' selbstständig und ohne Benutzung anderer als der angegebenen 
Hilfsmittel angefertigt habe; die aus fremden Quellen direkt oder indirekt übernommenen Gedanken sind als solche kenntlich gemacht. 
Die Arbeit wurde bisher in gleicher oder ähnlicher Form keiner anderen Prüfungskommission vorgelegt und auch nicht veröffentlicht.
% Etwas Abstand für die Unterschrift
\vspace{3cm}
% Hier kommt die Unterschrift drüber
\begin{tabbing}
\hspace{6cm}  \= \kill
\textit{Bremen, \today} \> \textit{Moritz Hanke}
\end{tabbing}

\end{appendix}

\end{document}


