\documentclass[12pt, a4paper]{report}
\usepackage{ucs}
\usepackage[T1]{fontenc}
\usepackage[utf8x]{inputenc}
\usepackage[ngerman]{babel}
\usepackage[pdftex]{graphicx}
\usepackage{a4wide}
\usepackage[onehalfspacing]{setspace}
\usepackage[left=2.5cm,right=2.5cm,top=2.5cm,bottom=2.5cm]{geometry}
\usepackage[marginal]{footmisc}
\usepackage{url}

\usepackage{longtable}

\usepackage{rotating}


%für die Kopfzeile
\usepackage{fancyhdr}
\pagestyle{fancy}
\fancyhf{}
\fancyhead[C]{\nouppercase{\leftmark}}
\fancyhead[R]{\thepage} 

%für römische Zahlen
\newcommand{\RM}[1]{\MakeUppercase{\romannumeral #1}}


\usepackage{multirow}
\usepackage{tabularx}

%für Mathe
\usepackage{amsmath}



\usepackage{apager}


\usepackage{float}
\restylefloat{figure}


% Für Lit.-Verzeichnis folgenden Befehl im Terminal ausführen: bibtex Masterarbeit_Moritz_Hanke
\bibliographystyle{apager_dgps}


% Fürs Abkürzungsverzeichnis folgenden Befehl im Terminal ausführen: makeindex Masterarbeit_Moritz_Hanke.nlo -s nomencl.ist -o Masterarbeit_Moritz_Hanke.nls
\usepackage{nomencl}
\let\abk\nomenclature
\renewcommand{\nomname}{Abkürzungsverzeichnis}
\setlength{\nomlabelwidth}{.20\hsize}
\renewcommand{\nomlabel}[1]{#1 \dotfill}
\setlength{\nomitemsep}{-\parsep}
\makenomenclature 


\usepackage{Sweave}



\begin{document}

% Titelblatt
\thispagestyle{empty}

\begin{center}
\textbf{\LARGE{TITEL DER ARBEIT\\}} 
\end{center}

\begin{center}
\textbf{\Large{\\Masterarbeit}}
\end{center}
\begin{verbatim}

\end{verbatim}

\begin{figure}[htbp]
\begin{center}
\setkeys{Gin}{width=0.5\textwidth}
\includegraphics{Universitaet_Bremen.png}
\end{center}
\end{figure}

\begin{center}
\textbf{Fachbereich 3: Mathematik \\
Studiengang Medical Biometry/Biostatistics (M.Sc.)\\}
\end{center}
\begin{verbatim}

\end{verbatim}

\begin{flushleft}
\begin{tabular}{lll}
& & \\
& & \\
\textbf{Eingereicht von:} & & Hanke, Moritz \\
\textbf{Geboren am:} & & 07.08.1985 \\
\textbf{Matrikelnummer:} & & 2404575 \\
& & \\
& & \\
\textbf{Betreuung:} & & Prof. Dr. Iris Pigeot\\
& & Dr. Ronja Foraita \\
& & \\
& & \\
\textbf{Eingereicht am:} & & \today\\
& & \\
& & \\
\end{tabular}
\end{flushleft}





%Verzeichnisse
\pagenumbering{Roman}
\tableofcontents

\listoffigures

\listoftables

\printnomenclature 

\abk{GWAS}{Genome-wide association study}



% Begin der Arbeit
\chapter{Einleitung}
\pagenumbering{arabic}
bla bla bla bla bla bla bla bla bla bla bla bla bla bla bla bla bla bla bla bla bla bla bla bla bla bla bla bla bla bla bla bla bla bla bla bla bla bla bla bla bla bla bla bla bla bla bla bla bla bla bla bla bla bla bla bla bla bla bla bla bla bla bla bla bla bla bla bla bla bla bla bla bla bla bla bla bla \cite{kim_network-based_2013}




\chapter{Theorie}
\section{Netzwerke und Graphen}
\subsection{Graphen}
Mit einem Graphen $G=(V,E)$ wird ein abstraktes Objekt bezeichnet, das aus einer Menge $V=\{v_1,\dots,v_{N_V}\}$ an \textit{Knoten} und einer Menge $E=\{e_1,\dots,e_{N_E}\}$ an \textit{Kanten} besteht \cite{brandes2005graphfunda}. Die Kardinalitäten $N_V =|N|$ und $N_E=|E|$ geben die \textit{Ordnung} und die \textit{Größe} des Graphen an. Ist eine Kante $e_k \in E$ die Zuordnung zweier Knoten $v_i,v_j \in V$ eines Graphen $G=(V,E)$, gilt ${v_i,v_j} \in e_k$. Die Knoten $v_i$ und $v_j$ werden \mbox{i.d.F.} als \textit{inzident} zu dem Knoten $e_k$ und \textit{adjazent} zueinander bezeichnet. Für $e_k=\{v_i,v_j\}$ handelt es sich um ein \textit{ungeordnetes} Paar an Knoten, d.h. $\{v_i,v_j\}=\{v_j,v_i\}$. Graphen, die nur ungeordneten Knotenpaaren beinhalten, sind \textit{ungerichtet}. Dagegen bezeichnet $e_k=(v_i,v_j)$, dass es sich um ein \textit{geordnetes} Paar an Knoten handelt und die Kante $e_k$ vom \textit{Anfangsknoten} $v_i$ auf den \textit{Endknoten} $v_j$ gerichtet ist, womit $(v_i,v_j) \neq (v_j, v_i)$ gilt \cite{kolaczyk2009statistical}. Als \textit{gerichtete} Graphen werden Graphen bezeichnet, die nur geordnete Knotenpaare beinhalten. Der Graph $H=(W,F)$ ist ein \textit{Subgraph} von $G=(V,E)$ wenn $W \subseteq V$ und $F \subseteq E$ gilt. Die in dieser Arbeit verwendeten Graphen sind \textit{schlichte} Graphen, d.h. es gibt weder \textit{Schlingen}\footnote{ Schlingen sind Kanten, die nur einen Endknoten an beiden Kantenenden besitzen, dh. $e_k=\{v_i,v_j\}$ bzw. $e_k=(v_i,v_j)$ für $i=j$.} noch mehrere Kanten zwischen zwei Knoten \cite{tittmann2011graphen}. Darüber hinaus werden nur \textit{endliche} Graphen berücksichtigt.\\
Der Knotengrad $d(v_i)$ für $v_i \in V$ gibt die Anzahl der Kanten an, die inzident auf den Knoten $v_i$ sind. Für einen Graphen $G=(V,E)$ sind der \textit{Minimumknotengrad} $\delta(G)$, der \textit{Maximumknotengrad} $\Delta(G)$ und der \textit{durchschnittliche Knotengrad} $d(G)$ als
\begin{align}
\delta(G):=\min\{d(v_i) \ | \ v_i \in V\}
\end{align}
\begin{align}
\Delta(G):=\max\{d(v_i) \ | \ v_i \in V\}
\end{align}
\begin{align}
d(G):=\frac{1}{N_V}\sum_{v_i \in V}d(v_i)
\end{align}
definiert \cite{diestel2006graph}.\\
Die \textit{Adjazenzmatrix} $\textbf{A}=(a_{i,j})_{N_V \times N_V}$, gibt an, ob es eine Kante zwischen zwei Knoten gibt und es sei 
\begin{align}
a_{i,j} := \begin{cases}
1, \qquad &\text{wenn} \ (v_i,v_j) \in E,\\
0, \qquad &\text{sonst \ .}\\
\end{cases}
\end{align}
Handelt es sich bei $G=(V,E)$ um einen ungerichteten Graphen ist $\textbf{A}$ eine symmetrische Matrix \cite{kolaczyk2009statistical}.\\
Die \textit{Inzidenzmatrix} $\textbf{\~B}=(b_{i,k})_{N_V \times N_E}$ eines \textit{ungerichteten} Graphen $G=(V,E)$ gibt an ob eine Kante $e_k \in E$ inzident zu $v_i \in V$ ist \cite{kolaczyk2009statistical} und es sei
%nicht gut, dass da schon wieder "sei" ist
\begin{align}
b_{i,k} := \begin{cases}
1, \qquad &\text{wenn} \ v_i \in e_k \ ,\\
0, \qquad &\text{sonst .}\\
\end{cases}
\end{align}
Für einen \textit{gerichteten} Graphen $G=(V,E)$ kann unterschieden werden ob $v_i \in e_k$ der Anfangs- oder Endknoten ist. \mbox{I.d.F} ist die Inzidenzmatrix $\textbf{B}=(b_{i,k})_{N_V \times N_E}$ als
\begin{align}
b_{i,k} := \begin{cases}
-1, \qquad &\text{wenn} \ (v_i,v_j) \in e_k \ ,\\
\ 1, \qquad &\text{wenn} \ (v_j,v_i) \in e_k \ ,\\
\ 0, \qquad &\text{sonst}\\
\end{cases}
\end{align}
definiert\footnote{Für die ersten beiden Fälle kann auch das Vorzeichen getauscht werden. Siehe dazu bspw. \citeNP{kolaczyk2009statistical}.} \cite{brandes2005graphfunda}.\\
Wird mit $\textbf{D}$ eine \textit{Diagonalmatrix} bezeichnet, deren $i,i$-Eintrag $d(v_i)$ entspricht, gilt $\textbf{BB}^T=\textbf{D}-\textbf{A}$, wobei $\textbf{B}^T$ die transponierte Matrix $\textbf{B}$ ist. $\textbf{BB}^T$ ist eine $N_V \times N_V$ Matrix und wird als \textit{Laplace-Matrix} $\textbf{L}$ bezeichnet \cite{kolaczyk2009statistical}. Als \textit{normalisierte Laplace-Matrix} wird $\textbf{\~L}$ bezeichnet, für die $\textbf{\~L}=\textbf{D}^{1/2}\textbf{L}\textbf{D}^{1/2}$ gilt. $\textbf{D}^{1/2}$ bezeichnet eine Diagonalmatrix, deren Eintrag in der $i,i$-Zelle $\frac{1}{\sqrt{d(v_i)}}$ bzw. $0$ wenn $d(v_i)=0$ für $v_i \in V$ ist.

%CLUSTER?
%Cliquen? (maximale?)
% induced subgraph
%E(G) oder E?
%DAG? <- kommt das bei Kim et al. vor?
%wird minimum, maximum, durchschnitsknotengrad benötigt?
%macht der zweite teil bei der definition von schlingen in der Fussnote Sinn bzgl. der gerichtetheit der Kante?

\bibliography{Masterarbeit}


\begin{appendix}
% Die eidesstattliche Erklärung auf einer neuen Seite
\newpage
% Keine Nummerierung für diesen Teil
\section*{Eigenständigkeitserklärung}
% Keine Kopf- und Fußzeilen ausgeben
\thispagestyle{empty}
% Aber trotzdem ins Inhaltsverzeichnis aufnehmen
\addcontentsline{toc}{chapter}{Eigenständigkeitserklärung}
% Hier der offizielle Text der eidesstattlichen Erklärung
Hiermit versichere ich, dass ich die vorliegende Arbeit "`XXXXXXXXXXXXXXXXXXXXXXXXXXXXXXXXX XXXXXXXXXXXXXXXXXXXXXXXXXXXXXXXXXXXXXXX XXXXXXXXXXXXXXXXXXXXXXXXXXXX XXXXXXXXXXXXXXXX"' selbstständig und ohne Benutzung anderer als der angegebenen 
Hilfsmittel angefertigt habe; die aus fremden Quellen direkt oder indirekt übernommenen Gedanken sind als solche kenntlich gemacht. 
Die Arbeit wurde bisher in gleicher oder ähnlicher Form keiner anderen Prüfungskommission vorgelegt und auch nicht veröffentlicht.
% Etwas Abstand für die Unterschrift
\vspace{3cm}
% Hier kommt die Unterschrift drüber
\begin{tabbing}
\hspace{6cm}  \= \kill
\textit{Bremen, \today} \> \textit{Moritz Hanke}
\end{tabbing}

\end{appendix}

\end{document}


